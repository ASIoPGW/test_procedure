% Chapter 1

\chapter{Test Procedure} % Main chapter title

\label{Chapter1} % For referencing the chapter elsewhere, use \ref{Chapter1} 

%----------------------------------------------------------------------------------------

\section{Introduction}
\section{Electronics test}
\section{Optical test}
\subsection{Mode matching and beam quality}
\section{Transfer function measurement}
\section{Noise measurement}
\subsection{Relative power noise}
\subsection{Harmonics noise}
\section{Injection test}
\red{
	I think the time-delay measurement by sine-gaussian signal should be categorized as "Calibration"\\
	Phase reconstruction ?
	\begin{itemize}
		\item Sine-gaussian
		\item CBC NR or EOB waveform
		\item Burst (CCSNe ?)
		\item CW \\	
			I don't understand the practical difficulty of continuous wave injection.
	\end{itemize}
}
\subsection{Hardware Property}
	\subsubsection{Anti-Imaging Chassis}
\red{Do we need de-whitening chassis to increase effective dynamical range of DGS output?}
	\subsubsection{OFS Low-Pass Filter Transfer Function}

\subsection{File preparation}
\red{How to take care of the inverse transfer function of hardware in each stage.}
	\subsubsection{Compact Binary Coalescence signal}
		\begin{itemize}
			\item BH-BH
			\item BH-NS
			\item NS-NS\\
				\red{the ability of reconstructing post-merger phase}
		\end{itemize}
	\subsubsection{Gravitational Wave Burst Signal}
		\begin{itemize}
			\item CCSNe
		\end{itemize}
	\subsubsection{Continuous Waves Signal}




\subsection{}
\section{Summary}

%----------------------------------------------------------------------------------------

