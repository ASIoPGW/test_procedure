% Chapter 1

\chapter{Test Procedure} % Main chapter title

\label{Chapter1} % For referencing the chapter elsewhere, use \ref{Chapter1} 

%----------------------------------------------------------------------------------------

\section{Introduction}
\section{Electronic test}
For the electronic tests, we test the performance of Optical Follower Servo chassis and Interface chassis using the test procedures from LIGO (DCC number T1400486-v2 and T1400484-v2).
\subsection{Optical Follower Servo board test}
\begin{enumerate}
	\item \textbf{Overview}\\
	The PCal Optical Follower Servo Chassis includes two Optical Follower Servo Boards, and two Optical Follower Back Board. The function of this chassis is to drive two AOMs with a whatever voltage is necessary in order to make the light coming out of the AOM match the sinusoidal shape of the excitation signal.
	\item \textbf{Test Equipment}
	\begin{enumerate}
		\item Power Supply capable of ~$\pm$18V,~2A
		\item Digital Multimeter (DMM)
		\item Power Supply
		\item Oscilloscope
		\item Function Generator
		\item Spectrum analyzer
	\end{enumerate}
	
	\item \textbf{DC test}\\
	\begin{enumerate}
		\item Turn on the power supply and the switches, and check the LED status below:
		\begin{center}
			\resizebox{13cm}{!}{
				\begin{tabular}{| c | c | c | c |}
					\hline
					\multirow{2}{*}{\textbf{Voltage}} & \multirow{2}{*}{\textbf{Current}}  & \multirow{2}{*}{\textbf{Observed Current}}  & \textbf{Front Panel LEDs On,}\\
					& & & \textbf{$\pm$15V and $\pm$12V?}\\ \hline
					+18V & 20mA$\pm$ 5mA &   & \\ \hline
					-18V & 19mA$\pm$ 5mA &   & \\
					\hline
				\end{tabular}
			}
		\end{center}
		\item Attach a standard 9-pin breakout board to the D-sub labeled “From PCal PD”. With a DMM, check the voltages on the pins in the table below:
		\begin{center}
			\resizebox{13cm}{!}{
				\begin{tabular}{| c | c | c |}
					\hline
					\textbf{Pins} & \textbf{Voltage Expected} & \textbf{Voltage Observed}\\ \hline
					J1 Pin2(+) and 7(GND) & +15V$\pm$ 0.5V & \\ \hline
					J1 Pin 3(-) and 7(GND) & -15V$\pm$ 0.5V &  \\
					\hline
				\end{tabular}
			}
		\end{center}
		
		\item \textbf{Loop Switch test:}\\
		Attach a 15-pin D-sub Breakout board to the “To/From PCal Interface” connector, and short together pins 4$\&$12. Read the voltage at pins 5(+) and 12(GND). You should read +12V $\pm$ 0.5V. At the same time, the green front panel LED labeled “Loop Closed” should illuminate.\\
		+12V present on pin 5?\underline{\qquad\qquad}\\
		“Loop Closed” LED lit?\underline{\qquad\qquad}\\
		
	\end{enumerate}
	\item \textbf{Functional Tests}
	\begin{enumerate}
		\item \textbf{PD Signal Tests:}\\
		PD Signal Tests: With the short for the loop switch still in place, it is possible to test the various signal outputs of the servo electronics. Using a voltage calibrator, put a negative (-) 4.8V level on the servo gain channel (From PCal Interface connector, Pin 1(-4.8V) and Pin 9 (GND)). This should give a gain of 1 (0dB) from the variable Gain amplifier. Input a 100mV signal from the network analyzer into the “From PCal PD” connector Pin 1(+) and Pin 6(-), and sweep from 100Hz to 100KHz. Measure the signals, and fill in the table below:
		\begin{center}
			\resizebox{13cm}{!}{
				\begin{tabular}{| c | c | c |}
					\hline
					\textbf{Output} & \textbf{Expected} & \textbf{Observed}\\ \hline
					"PD Mon" BNC & 0dB +/- 0.5dB Flat & \\ \hline
					"Err Mon" BNC & 0dB +/- 0.5dB Flat &  \\ \hline
					“To/From PCal Interface” & 0dB +/- 0.5dB Flat & \\
					Pins 6(+) and 14(-) & read differentially(A-B)  &\\ \hline
					“To/From PCal Interface” & 0dB +/- 0.5dB Flat & \\
					Pins 7(+) and 15(-) & read differentially(A-B)  &\\ \hline
					\multirow{2}{*}{“Out Mon” BNC} & -8.2dB at DC with 2 poles @ & \\
					& 3KHz, and 1 zero @30KHz &\\ \hline
					“Out Mon” BNC & -90º of phase at 3KHz +/- 2º &\\ \hline
					\multirow{2}{*}{“Out Mon” BNC} & Rising to -125.5º of phase @   &\\ 
					& 30KHz $\pm$ 2º & \\ \hline
					\multirow{2}{*}{“To AOM” BNC} & -8.2dB at DC with 2 poles @ & \\
					& 3KHz, and 1 zero @30KHz &\\ \hline
					“To AOM” BNC & -90º of phase at 3KHz +/- 2º &\\ \hline
					\multirow{2}{*}{“To AOM” BNC} & Rising to -125.5º of phase @   &\\ 
					& 30KHz $\pm$ 2º & \\ 	
					\hline
				\end{tabular}
			}
		\end{center}
		
		\item \textbf{Excitation Signal test:} \\
		Move the input signal to the appropriate connector below, and read from the “To AOM” BNC. Record the results in the table below:
		\begin{center}
			\resizebox{13cm}{!}{
				\begin{tabular}{| c | c | c | c |}
					\hline
					\textbf{Input} & \textbf{Output} & \textbf{Expected} & \textbf{Observed}\\ \hline
					“From DAC”  & \multirow{3}{*}{“To AOM” BNC} & -8.2dB at DC with 2 & \\
					connector Pins & & poles @ 3KHz, and 1 &\\
					1(+) and 6(GND) & & zero@30KHz & \\ \hline
					\multirow{3}{*}{“CLTF Test In” BNC}  & \multirow{3}{*}{“To AOM” BNC} & -8.2dB at DC with 2 & \\
					& & poles @ 3KHz, and 1 &\\
					& & zero@30KHz & \\ 
					\hline
				\end{tabular}
			}
		\end{center}
		\item \textbf{Gain tests:} \\
		With the sine wave generator, input an appropriate amplitude 500Hz sine wave into the “From PCal PD” connector, Pins 1(+) and 6(-) Watch the output signal on the “To AOM” BNC connector on an oscilloscope. Vary the gain voltage level from the voltage calibrator from negative (-)9.6V to positive(+)9.6V (From PCal Interface connector, Pin 1(-4.8V) and Pin 9 (GND)). The visible gain should vary from 0.143V/V to 124 V/V (-14dB to 46dB), Record the results in the table below, either by measuring the p-p signal with cursors, or getting the RMS amplitude from the scope’s “measure” function:
		\begin{center}
			\resizebox{13cm}{!}{
				\begin{tabular}{| c | c | c | c |}
					\hline
					\textbf{Input signal level} & \textbf{Input Gain Level} & \multirow{3}{*}{\textbf{Expected Output}} & \multirow{3}{*}{\textbf{Observed Output}}\\
					\textbf{“From PCal PD” }& \textbf{“To/From PCal Interface”} & & \\
					\textbf{1(+) and 6(-)} & \textbf{Pin1 and Pin 9(GND)} & & \\ \hline
					\multirow{2}{*}{1$\mbox{V}_{\mbox{p-p}}$}  & \multirow{2}{*}{-9.6V} & 143 $\mbox{mV}_{\mbox{p-p}}$ $\pm$ 10mV& \\
					& &  or 55 $\mbox{mV}_{\mbox{rms}}$ $\pm$ 10mV &\\ \hline
					\multirow{2}{*}{1$\mbox{V}_{\mbox{p-p}}$}  & \multirow{2}{*}{0V} & 4.12 $\mbox{V}_{\mbox{p-p}}$ $\pm$ 100mV& \\
					& &  or 1.48 $\mbox{mV}_{\mbox{rms}}$ $\pm$ 10mV &\\ \hline
					\multirow{2}{*}{0.03$\mbox{V}_{\mbox{p-p}}$}  & \multirow{2}{*}{+9.6V} & 3.72 $\mbox{mV}_{\mbox{p-p}}$ $\pm$ 10mV& \\
					& &  or 1.43 $\mbox{mV}_{\mbox{rms}}$ $\pm$ 10mV &\\ 
					\hline
				\end{tabular}
			}
		\end{center}
		
		\item \textbf{Offset Input Test:}\\
		With the gain level set to negative (-)1V, Input a 0.1Vp-p, 500Hz Sine wave into the “From PCal PD” connector, Pins 1(+) and 6(-). Next, put a voltage level into the Offset channel, “To/From PCal Interface, Pins 2(+) and 10(-). There should be a gain of ~4.4V/V offset on the observed sine wave on the oscilloscope. Verify this in the table below:
		\begin{center}
			\resizebox{13cm}{!}{
				\begin{tabular}{| c | c | c |}
					\hline
					\textbf{Offset Input} & \textbf{Offset Expected} & \textbf{Offset Observed}\\ \hline
					0V & 0V$\pm$ 0.2V & \\ \hline
					1V &  1.5V$\pm$ 0.2V &  \\ \hline
					2V & 3V$\pm$ 0.2V &  \\
					\hline
				\end{tabular}
			}
		\end{center}
		\item \textbf{Oscillation Monitor tests:}\\ Put a negative (-) 4.8V level on the servo gain channel (From PCal Interface connector, Pin 1(-4.8V) and Pin 9 (GND)). Place a 1Vp-p, sine wave into the “From PCal PD” connector, Pins 1(+) and 6(-). Read the voltage at the “To/From PCal Interface” connector, Pins 3(+) and 11(-) with a DMM. Vary the frequency, and record the results in the table below:
		\begin{center}
			\resizebox{13cm}{!}{
				\begin{tabular}{| c | c | c |}
					\hline
					\textbf{Input Frequency} & \textbf{Output Expected} & \textbf{Output Observed}\\ \hline
					100Hz & & \\ \hline
					1KHz &   &  \\ \hline
					100KHz &  &  \\
					\hline
				\end{tabular}
			}
		\end{center}
	\end{enumerate}
\end{enumerate}
\subsection{Interface board test}
\section{Optical test}
\subsection{Mode matching and beam quality}
\section{Transfer function measurement}
\section{Noise measurement}
\subsection{Relative power noise}
\subsection{Harmonics noise}
\section{Injection test}
\red{
	I think the time-delay measurement by sine-gaussian signal should be categorized as "Calibration"\\
	Phase reconstruction ?
	\begin{itemize}
		\item Sine-gaussian
		\item CBC NR or EOB waveform
		\item Burst (CCSNe ?)
		\item CW \\	
			I don't understand the practical difficulty of continuous wave injection.
	\end{itemize}
}
\subsection{Hardware Property}
	\subsubsection{Anti-Imaging Chassis}
\red{Do we need de-whitening chassis to increase effective dynamical range of DGS output?}
	\subsubsection{OFS Low-Pass Filter Transfer Function}

\subsection{File preparation}
\red{How to take care of the inverse transfer function of hardware in each stage.}
	\subsubsection{Compact Binary Coalescence signal}
		\begin{itemize}
			\item BH-BH
			\item BH-NS
			\item NS-NS\\
				\red{the ability of reconstructing post-merger phase}
		\end{itemize}
	\subsubsection{Gravitational Wave Burst Signal}
		\begin{itemize}
			\item CCSNe
		\end{itemize}
	\subsubsection{Continuous Waves Signal}




\subsection{}
\section{Summary}

%----------------------------------------------------------------------------------------

